\documentclass[a4paper,12pt,titlepage]{article}

\usepackage[english]{babel}
\usepackage{amsmath}
\usepackage{amsfonts}
\usepackage{amssymb}
\usepackage{epsfig}
\usepackage{afterpage}
\usepackage{booktabs}
\usepackage{graphicx}
\usepackage{subfig}
\usepackage{graphicx}
\usepackage{subfig} 
\usepackage{todonotes}
\usepackage[framed,autolinebreaks,numbered]{mcode}
\usepackage[utf8]{inputenc} 
\usepackage{hyperref} %dit pakket gebruik je voor hyperlinks
\usepackage[titletoc,title,page,header]{appendix}


%package en settings voor symbolenlijst
\usepackage{nomencl}
\makenomenclature
\renewcommand{\nomname}{Symbolenlijst}

%package en settings voor headers/footers
\usepackage{fancyhdr}
\pagestyle{fancy}
\setlength{\headheight}{14.5pt}
\fancyhf{}

%voeg commando toe voor lege pagina's i.v.m. printen
\newcommand*\emptypage{\newpage\null\thispagestyle{empty}\newpage}

%commando voor lijnen op titelpagina
\newcommand{\HRule}{\rule{\linewidth}{0.5mm}}

\newcommand{\thesectionabstract}{Abstract}

\begin{document}
\begin{titlepage}
\begin{center}

% Upper part of the page. The '~' is needed because \\
% only works if a paragraph has started.

%naam universiteit en naam Vak plaatsen
\textsc{\LARGE University of Twente}\\[1.5cm]

\textsc{\Large Course}\\[0.5cm]

% Titel plaatsen
\HRule \\[0.4cm]
{ \huge \bfseries Title}\\[0.4cm]

%afbeelding toevoegen
\HRule \\[1cm]
\includegraphics[scale=0.3]{figures/muzieknoot}
\HRule \\[2cm]

% Auteurs plaatsen
\begin{minipage}{0.4\textwidth}
\begin{flushleft} \large
\emph{Authors:}\\
Bas \textsc{Stadhouder}\\s1219448
\end{flushleft}
\end{minipage}
\begin{minipage}{0.4\textwidth}
\begin{flushright} \large
\emph{} \\
Bas \textsc{Stadhouder}\\s1219448
\end{flushright}
\end{minipage}

\vfill

% Datum plaatsen
%\begin{minipage}{1.0\textwidth}
%\begin{flushleft} \large
{\large \today}
%\end{flushleft}
%\end{minipage}

\end{center}
\end{titlepage}


%Headers and footers maken
\renewcommand{\sectionmark}[1]{\markright{\thesection\ #1}}
\fancyhead[LO]{\textbf{Course}}
\fancyhead[RO]{\nouppercase{\bfseries\rightmark}}
\lfoot{Bas Stadhouder}
\rfoot{\textbf{\thepage}}
\renewcommand{\footrulewidth}{0.4pt}



\emptypage
%\section*{Empty page for printing}
%use \textbackslash emptypage command if needed otherwise fill it
%\newpage

\section*{Abstract} \markright{Abstract}
\addcontentsline{toc}{section}{\thesectionabstract}

\emptypage

%inhoudsopgave plaatsen
\tableofcontents

\newpage
\emptypage



\section{Introduction}
\newpage
\section{Theory}
\section{Measurements}



\newpage
\markright{Appendices}
\begin{appendices}
\section{Matlab Code}
\lstinputlisting{matlab/tunneling.M}
\end{appendices}
\bibliography{refs/standaard}{}


\end{document}
